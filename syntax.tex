\section{syntax and semantics}
\label{sec:syntax}
\begin{itemize}
	\item talk about the actual syntax \lstinline{<foo>3}
	\begin{itemize}
		\item The angle braces `<` are not used. so are free to use
		\item the biggest problem with this syntax is that there is no way of telling where the levenstien distence token ends. - consider syntax \lstinline{<foo>{3}}.
		\item alternatively leave it, as there is always the workaround of using a single element charicter class to terminate the token. (Impotance of unambigous grammar?)
	\end{itemize}
	\item Ability to specify optional non-greedyness
	\item greedyness - pick longest
	\item matching closness - pick closest, prioatise over length.
	\item interaction with other operators - rematch on repation.
\end{itemize}
