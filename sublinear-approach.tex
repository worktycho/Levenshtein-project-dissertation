\section{Sublinear Approach}
\label{sec:sublinear}

\begin{itemize}
	\item The sublinear approach is based on the ability to construct suffix trees in linear time. The paper uses McCreights algorithm, but the Ukkonen algorthm may be preferable.
	\item The sublinear approach uses the suffix trees to work out sections of text that can match the pattern.
	\item This approach provides linear time matching amortized, and significant amounts of the work are only dependent on the pattern.
	\item The pattern is compiled to a set of matching statistics.
	\item These charicteristics are preferable for regexs because it enables easy integration with a backtracking regex algorithm.
	\item The regex engine will identify positions that the regex chunk preceding the pattern can match to, which this algorithm can then match or not.
	\item This requires an algorithm that can rapidly match from multiple non-contigous starting positions in a string.
	\item When the levenstien pattern starts the whole pattern, a matrix based whole string matching algorithm may be more apropriate, but that is O(mn), so may be slower.
\end{itemize}
\cite{Chang1994}
